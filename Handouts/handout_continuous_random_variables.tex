\documentclass[11pt]{article}
\usepackage[utf8]{inputenc}
\usepackage{amsmath}
\usepackage{amssymb}
\usepackage{graphicx}
\usepackage{indentfirst}
\usepackage{hyperref}
\let\oldemptyset\emptyset
\let\emptyset\varnothing


\title{\textbf{Esssentials of Applied Data Analysis\\
				IPSA-USP Summer School 2017}\newline\\
				Handout - Continous random variables}

\author{Leonardo Sangali Barone\\ \href{leonardo.barone@usp.br}{leonardo.barone@usp.br}}
\date{jan/17}

\begin{document}

\maketitle

\section*{Continuous random variables}
	
	The rules that apply to discrete random variables also apply to continous random variables.\\
	
	The main problem when we deal with a continuous random variable is that we cannot count every possible outcome and multiply it by the probability of that outcome occurin (remember: continous variables are and infinite set and uncontable!)\\

	\subsection*{Expectation and variance of a continuous random variable}

	In other words, we cannot do this:
	
	\[\sum\limits_{i=1}^n P(X = x_i) = \sum\limits_{i=1}^n f(x_i) = 1\]
or this
		\[E[X] = \sum\limits_{i=1}^n x_i* P(X=x_i) =\sum\limits_{i=1}^n x_i * f(x_i)\]
or this
		\[Var[X] = \sum\limits_{i=1}^n[x_i - E[X]]^2 * P(X=x_i) =\sum\limits_{i=1}^n[x_i - E[X]]^2 * f(x_i)\]
		
		because can't count every $x_i$. What can we do instead?\\

	We can use integrals (see Moore and Siegel, chap 7 for integrals) to sum the area of the continuous distributions and then calculate the expected value and variance:
	
	\[\int_{-\infty}^{\infty} \text{f(x) dx} = 1\]
	\[E[X] = \int_{-\infty}^{\infty}  \text{x f(x) dx}\]
	\[Var[X] = \int_{-\infty}^{\infty}  [x - E[X]]^2 \text{f(x) dx}\]


\end{document}
