\documentclass[11pt]{article}
\usepackage[utf8]{inputenc}
\usepackage{amsmath}
\usepackage{amssymb}
\usepackage{graphicx}
\usepackage{indentfirst}
\usepackage{hyperref}
\let\oldemptyset\emptyset
\let\emptyset\varnothing


\title{\textbf{Esssentials of Applied Data Analysis\\
				IPSA-USP Summer School 2017}\newline\\
				Handout - The Basics of Probability Theory - II}

\author{Leonardo Sangali Barone\\ \href{leonardo.barone@usp.br}{leonardo.barone@usp.br}}
\date{jan/17}

\begin{document}

\maketitle

\section*{Introduction to Probability - Part II}

	Basic Notions of probability, part II.

	\subsection*{Single or Compound events? Independence and Exclusivity}

	An event can be simple (a single outcome) or compound (two or more single events).\\

	The relation between the events that form a compound event can be defined as:
	
	\begin{itemize}
		\item \emph{Independent}: two events are independent if the probability that one occurs does not change as a consequence of the other event’s occurring.
		\item \emph{Mutual exclusivity}: two events are mutually exclusive when one cannot occur if the  other has occurred.
		\item \emph{Collective exhaustivity}: the set of collective exhaustive events is the whole sample space.
	\end {itemize}


	\subsection*{Axioms and theorems of probability (2)}
	\begin{itemize}
		\item  If $A$ and $B$ are mutually exclusive, then:
	\[P(A \cup B) = P(A) + P (B)\]
		\item  If $A_1$, $A_2$, ... is a sequence of mutually exclusive events, then:
	\[P(A_1 \cup A_2 \cup ... \cup A_n) = P(A_1) + P (A_2) + ... + P(A_n)\]
	\end{itemize}


	\subsection*{Dices and mutually exclusive events}

	Roll a 6-side dice.\\
	
	What's is the probability of getting a 5 \textbf{OR} a 6?\\
	\[P(\text{5 or 6}) = P(5 \cup 6) = ?\]\\
	
	What's is the probability of getting a prime \textbf{OR} an odd number?\\
	\[P(\text{prime or odd}) = P(\text{prime }\cup\text{ odd}) = ?\]

	\subsection*{Dices and mutually exclusive events - answers}

	Roll a 6-side dice.\\
	
	What's is the probability of getting a 5 \textbf{OR} a 6?\\
	\[P(\text{5 or 6}) = P(5 \cup 6) = \frac{\#\{5, 6\}}{6} = \frac{2}{6} = \frac{1}{3} = \frac{1}{6} + \frac{1}{6} = P(5) + P(6)\]\\ 
	since 5 and 6 are mutually exclusive events.\\
	
	What's is the probability of getting a prime \textbf{OR} an odd number?\\

	\textbf{Beware!!!} This is not true for events that are not mutually exclusive (note that 3 and 5 are both prime and odd).

	\[P(\text{prime}) + P(\text{odd}) = \frac{\#\{2, 3, 5\}}{6} + \frac{\#\{1, 3, 5\}}{6} = \frac{3}{6} + \frac{3}{6} = \frac{6}{6} = 1\]
	
	\[P(\text{prime or odd}) = P(\text{prime }\cup\text{ odd}) = \frac{\#\{1, 2, 3, 5\}}{6} = \frac{4}{6} \neq P(\text{prime}) + P(\text{odd})\]


	\subsection*{Mutually exclusive events - random legislator}

	Let's go back to the Legislative House example.\\
	
	What's is the probability of getting a legislator from Party A \textbf{OR} B?
	\[P(\text{A or B}) = P(A \cup B) = ?\]
	
	What's is the probability of getting a legislator from Party A \textbf{OR} a woman (W)?
	\[P(\text{A or W}) = P(A \cup W) =?\]


	\subsection*{Mutually exclusive events - random legislator - answers}
	
	Let's go back to the Legislative House example.\\
	
	What's is the probability of getting a legislator from Party A \textbf{OR} B?
	\[P(\text{A or B}) = P(A \cup B) = P(A) + P(B)\]
	
	What's is the probability of getting a legislator from Party A \textbf{OR} a woman (W)?\\
	
	If there is at leat a woman on party A, the events are not mutually excluise and \[P(\text{A or W}) = P(A \cup W) \neq P(A) + P(W)\]
	
	If it is a all-men party, being a woman and beloging to party A are mutually exclusive and
	\[P(A \cup W) = P(A) + P(W)\]


	\subsection*{Axioms and theorems of probability (3)}
	$A^c$ is the complementary event of A.
	\begin{itemize}
		\item  $P(A^c) = 1 - P(A)$
		\item  $P(A \cup A^c) = P(A) + P(A^c) = 1$ (because they are mutually exclusive)
	\end{itemize}


	\subsection*{Complementary event (not A, ~A, A' or $A^c$)}
	What is the probability of \textbf{NOT} getting a 5 on a 6-side dice?
	\[P(\text{not 5}) = P\text{(\{1,2,3,4,6\})} = \frac{\#\{1, 2, 3, 4, 6\}}{6} =  5/6	\]

	Or, more elegantly:
	\[P(\text{not 5}) = 1 - P(5) = 1 - \frac{1}{6} = \frac{5}{6}\]

	\subsection*{Axioms and theorems of probability (4)}
	\begin{itemize}
		\item $P(A \cap B)$ is the the probability of $A$ \textbf{AND} $B$ happening at the same time.																																																																																																																		
		\item  $P(A \cup B) = P(A) + P(B) - P(A \cap B)$ 
	\end{itemize}

	\subsection*{Not mutually exclusive events - dice and legislators}
	Going back to our problem, what's is the probability of getting a prime \textbf{OR} an odd number?
	
	\[P(\text{prime }\cup\text{ odd}) = P(\text{prime}) + P(\text{odd}) - P(\text{prime }\cap\text{ even}) 
	= \frac{3}{6} + \frac{3}{6} - \frac{2}{6} = \frac{4}{6} = \frac{2}{3}\]
	
	What's is the probability of getting a legislator from Party A \textbf{OR} a woman (W)?\\
	
	If there is at leat a W on party A, then
	\[P(A \cup W) = P(A) + P(W) - P(A \cap W)\]

\end{document}