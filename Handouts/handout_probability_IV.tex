\documentclass[11pt]{article}
\usepackage[utf8]{inputenc}
\usepackage{amsmath}
\usepackage{amssymb}
\usepackage{graphicx}
\usepackage{indentfirst}
\usepackage{hyperref}
\let\oldemptyset\emptyset
\let\emptyset\varnothing


\title{\textbf{Esssentials of Applied Data Analysis\\
				IPSA-USP Summer School 2017}\newline\\
				Handout - The Basics of Set Theory}

\author{Leonardo Sangali Barone\\ \href{leonardo.barone@usp.br}{leonardo.barone@usp.br}}
\date{jan/17}

\begin{document}

\maketitle

\section*{Colored balls, boxes and political parties.}


	\subsection*{Bayes' rule}
	Suppose there are $A_1$, $A_2$,..., $A_n$ events and $A_1 \cup A_2 \cup... \cup A_n$ = S. $A_i$ events are mutually exclusive. Suppose there is another event $B$. For each $A_i$, $B$ has a different joint conditional probability $P(A_i|B)$. $B$ can be described as:
	
	\[P(B) = P(A_1)*P(B|A_1)+P(A_2)*P(B|A_2)+...+P(A_n)*P(B|A_n)\]
 
	The probability that an specific $A_i$ will occur given B is defined as:

	\[P(A_i|B) = \frac{P(A_i)*P(B|A_i)}{P(A_1)*P(B|A_1)+P(A_2)*P(B|A_2)+...+P(A_n)*P(B|A_n)}\]
		


	\subsection*{Bayes' rule - a simple example}
	There are two political parties, $A_1$, with 60 members and $A_2$, with 40 members. All members of party $A_1$ are against the country engaging in a new war. In party $A_2$, one third of the members are against the new war, but the others are in favor of it. $B$ is the event of being against enganging in war.
	 
	You turn on the radio and there is a politician speaching against war. You don't know her and you want to figure out her party. What is the probability that she belongs to party $A_2$ given she is against war?
		


	\subsection*{Bayes' rule - a simple example}
	Let's think about the problem for a second. Whithout any calculations, is she more likely to belong to which party? Make your guess.
	
	What is the probability that a a politician at random belongs to party $A_1$? And party $A_2$?
	\[P(A_1) = 40/100 = 2/5; P(A_2) = 1 - P(A_1) = 1 - 2/5 = 3/5;\]
	So without knowing that she is against the war, it is more likely that she belgons to party $A_2$, right? 3/5 would be a good first guess. But we do know she is against war and we can \textbf{update} our "guess" with this information.
	


	\subsection*{Bayes' rule - a simple example}
	What is the probability that a politician from $A_1$ is against war? $P(B|A_1) = 1$. And from $A_2$? $P(B|A_2) = 1/3$.
	
	Since we know that being or not in favor of war is conditioned on the party, we can \textbf{update} our guess in that way.
	
	\[P(A_2|B) = \frac{P(A_2)*P(B|A_2)}{P(A_1)*P(B|A_1)+P(A_2)*P(B|A_2)}=\]
	\[= \frac{3/5*1/3}{2/5*1+3/5*1/3} = \frac{1}{3} < P(A_2) = 3/5\]	
	


	\subsection*{Bayes' rule - colored balls}
	There are 3 boxes, each containing 10 colored balls. In the first box ($A_1$) there are 8 blue balls and 2 red balls. In the second box ($A_2$) there are 3 blue balls and 7 red balls. There are only red balls in the third box ($A_3$) I have randomly chosen a box and pick up randomly one ball from it. The ball was blue ($Blue$). What is the probability that I got the first box ($A_1$)?


	\subsection*{Bayes' rule - colored balls}
	There are 3 boxes, each containing 10 colored balls. In the first box ($A_1$) there are 8 blue balls and 2 red balls. In the second box ($A_2$) there are 3 blue balls and 7 red balls. There are only red balls in the third box ($A_3$) I have randomly chosen a box and pick up randomly one ball from it. The ball was blue ($Blue$). What is the probability that I got the first box ($A_1$)?
	
	\small{\[P(A_1|B) = \frac{P(A_1)*P(Blue|A_1)}{P(A_1)*P(Blue|A_1)+P(A_2)*P(Blue|A_2)+P(A_3)*P(Blue|A_3)}\]}
	\[= \frac{1/3*8/10}{1/3*8/10+1/3*3/10+1/3*0} = \frac{8/30}{8/30+3/30+0} = \frac{8}{11} = 0.73\] 
	


	\subsection*{Bayes' rule - colored balls}
	There are 3 boxes, each containing 10 colored balls. In the first box ($A_1$) there are 8 blue balls and 2 red balls. In the second box ($A_2$) there are 3 blue balls and 7 red balls. There are only red balls in the third box ($A_3$) I have randomly chosen a box and pick up randomly one ball from it. The ball was blue ($Blue$). What is the probability that I got the first box ($A_1$)?


	\subsection*{Bayes' rule - positive medical exam}

	
	\small{\[P(A_1|B) = \frac{P(A_1)*P(Blue|A_1)}{P(A_1)*P(Blue|A_1)+P(A_2)*P(Blue|A_2)+P(A_3)*P(Blue|A_3)}\]}
	\[= \frac{1/3*8/10}{1/3*8/10+1/3*3/10+1/3*0} = \frac{8/30}{8/30+3/30+0} = \frac{8}{11} = 0.73\] 
	

	\subsection*{Bayes' rule - another way to think}
	A theorem that explains how one ought to change one’s subjective probability in response to empirical evidence. 
	
	Assume: \\
	T = theory or hypothesis \\
	E = represents a new piece of evidence that seems to confirm or 
	disconfirm the theory. 
	
	P(T)=  Prior Belief = Probability of (T) 
	
	Then, the Posterior Probability of T:
	\[P(T|E) = \frac{P(T)*P(E|T)}{P(T)*P(E|T)+P(T^c)*P(E|T^c)}=\]


	\subsection*{Bayes' rule - another way to think}
	A theorem that explains how one ought to change one’s subjective probability in response to empirical evidence. 
	
	Assume: \\
	T = theory or hypothesis \\
	E = represents a new piece of evidence that seems to confirm or 
	disconfirm the theory. 
	
	P(T)=  Prior Belief = Probability of (T) 
	
	Then, the Posterior Probability of T:
	\[P(T|E) = \frac{P(T)*P(E|T)}{P(T)*P(E|T)+P(T^c)*P(E|T^c)}=\]



	\subsection*{Bayes' rule - textbook example}
	Stokes campaing example - Moore and Siegel, chap 9, pp 188-190
	
\end{document}
