\documentclass[11pt]{article}
\usepackage[utf8]{inputenc}
\usepackage{amsmath}
\usepackage{amssymb}
\usepackage{graphicx}
\usepackage{hyperref}
\usepackage[parfill]{parskip}
\let\oldemptyset\emptyset
\let\emptyset\varnothing


\title{\textbf{Esssentials of Applied Data Analysis\\
				IPSA-USP Summer School 2017}}

\author{Leonardo Sangali Barone\\ \href{leonardo.barone@usp.br}{leonardo.barone@usp.br}}
\date{jan/17}

\begin{document}

\maketitle

\section*{Axioms and theorems of probability - summary}

	(Also know as ``rules of probability'')

	\begin{enumerate}
		\item For every event A, $0 \leq P(A) \leq 1$
		\item $P(S) = 1$
		\item $P(\emptyset) = 0$
		\item  If $A$ and $B$ are mutually exclusive, then:
	\[P(A \cup B) = P(A) + P (B)\]
		\item  If $A_1$, $A_2$, ... is a sequence of mutually exclusive events, then:
	\[P(A_1 \cup A_2 \cup ... \cup A_n) = P(A_1) + P (A_2) + ... + P(A_n)\]
		\item  $P(A^c) = 1 - P(A)$
		\item  $P(A \cup A^c) = P(A) + P(A^c) = 1$ (because they are mutually exclusive)
		\item $P(A \cap B)$ is the the probability of $A$ \textbf{AND} $B$ happening at the same time.																																																																																																																		
		\item  $P(A \cup B) = P(A) + P(B) - P(A \cap B)$
		\item If A and B are independent events, then $P(A \cap B) = P(A) * P(B)$
		\item If A and B are not independent events, then the conditional probability of A given B, $P(A|B)$, 	is defined as
	\[P(A|B) = \frac{P(A\cap B)}{P(B)}\]
		\item If A and B are independent events, then $P(A|B) = P(A)$ and the formula
	\[P(A|B) = \frac{P(A\cap B)}{P(B)}\]
	can be rewritten as (as previously seen)
	\[P(A\cap B) = P(A)*P(B)\]
		\item If A and B are mutually exclusive, then
		\[P(A\cap B) = 0\]
			hence
		 \[P(A|B) = P(B|A) = 0\] 

	\end{enumerate}

\end{document}