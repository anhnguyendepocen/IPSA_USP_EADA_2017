\documentclass[11pt]{article}
\usepackage[utf8]{inputenc}
\usepackage{amsmath}
\usepackage{amssymb}
\usepackage{graphicx}
\usepackage{hyperref}
\usepackage[parfill]{parskip}
\let\oldemptyset\emptyset
\let\emptyset\varnothing


\title{\textbf{Esssentials of Applied Data Analysis\\
				IPSA-USP Summer School 2017}\newline\\
				The Basics of Set Theory - Bayes' rule}

\author{Leonardo Sangali Barone\\ \href{leonardo.barone@usp.br}{leonardo.barone@usp.br}}
\date{jan/17}

\begin{document}

\maketitle

\section*{Bayes' Rule}

	Suppose there are $A_1$, $A_2$,..., $A_n$ events and $A_1 \cup A_2 \cup... \cup A_n$ = S. $A_i$ events are mutually exclusive. Suppose there is another event $B$. For each $A_i$, $B$ has a different joint conditional probability $P(A_i|B)$. $B$ can be described as:
	
	\[P(B) = P(A_1)*P(B|A_1)+P(A_2)*P(B|A_2)+...+P(A_n)*P(B|A_n)\]
 
	The probability that an specific $A_i$ will occur given B is defined as:

	\[P(A_i|B) = \frac{P(A_i)*P(B|A_i)}{P(A_1)*P(B|A_1)+P(A_2)*P(B|A_2)+...+P(A_n)*P(B|A_n)}\]
		
	\subsection*{Bayes' rule - a first example}
	
	There are two political parties, $A_1$, with 60 members and $A_2$, with 40 members. $80\%$ of party $A_1$ members are against the country engaging in a new war. In party $A_2$, one third of the members are against the new war, but the others are in favor of it. $nW$ is the event of being \textbf{against} enganging in war.

	You turn on the radio and there is a politician speaching against war. You don't know her and you want to figure out her party. What is the probability that she belongs to party $A_1$ given she is against war? Or party $A_2$?
		
	Let's think about the problem for a second. Whithout any calculations, is she more likely to belong to which party? Make your guess.
	
	What is the probability that a a politician at random belongs to party $A_1$? And party $A_2$?
	
	\[P(A_1) = \frac{40}{100} = \frac{2}{5} \]
	\[P(A_2) = \frac{60}{100} = \frac{3}{5} \]
Or
	\[P(A_2) = 1 - P(A_1) = 1 - \frac{2}{5} = \frac{3}{5}\]
	
	So without knowing that she is against the war, it is more likely that she belgons to party $A_1$, right? $2/5$ for $A_1$ and $3/5$ for $A_2$ would be a good first guess. But we do know she is against war and we can \textbf{update} our ``guess'' based on this information.

	What is the probability that a politician from $A_1$ is against war? 
	
	\[P(nW|A_1) =  \frac{4}{5} = 0.8\]
	
	And from $A_2$?
	
	\[P(nW|A_2) = \frac{1}{3}\]
	
	Since we know that being or not in favor of war is conditioned on the party, we can update our guess using Baye's rule.
	
	\[P(A_1|nW) = \frac{P(A_1)*P(nW|A_1)}{P(A_1)*P(nW|A_1)+P(A_2)*P(nW|A_2)} 
	= \frac{\frac{3}{5} * \frac{4}{5}}{\frac{3}{5} * \frac{4}{5} + \frac{2}{5} * \frac{1}{3}} =\]
	\[= \frac{\frac{12}{25}}{\frac{12}{25} + \frac{2}{15}} = \frac{\frac{12}{25}}{\frac{36+10}{75}} = \frac{\frac{12}{25}}{\frac{46}{75}} = \frac{12}{25} * \frac{75}{46} = \frac{36}{46} = \frac{18}{23} \simeq 0.7826 > P(A_1) = 0.6\]	
	
	So $P(A_1|nW) > P(A_1)$, our initial guess (which makes sense, since most of party $A_1$ members are against war). Let's look at what happens with $A_2$:

	\[P(A_2|nW) = \frac{P(A_2)*P(nW|A_2)}{P(A_1)*P(nW|A_1)+P(A_2)*P(nW|A_2)} 
	= \frac{\frac{2}{5} * \frac{1}{3}}{\frac{3}{5} * \frac{4}{5} + \frac{2}{5} * \frac{1}{3}} =\]
	\[= \frac{\frac{2}{15}}{\frac{12}{25} + \frac{2}{15}} = \frac{\frac{2}{15}}{\frac{36+10}{75}} = \frac{\frac{2}{15}}{\frac{46}{75}} = \frac{2}{15} * \frac{75}{46} = \frac{10}{46} = \frac{5}{23} \simeq 0.2173 < P(A_2) = 0.4\]	

	As expected, $P(A_2|nW) < P(A_2)$, since the minority of party $A_2$ is against war and the politian was speaching against war.
	
	Before we move on, think or a second about what we just did. We wanted to guess the political party of a politician and we had an expectation about the distribution of politicians within political parties ($A_1 = 0.6$ and $A_2 = 0.4$). We collected new information (data!!!!) about this specific politician and we updated our believes about what political parties she \emph{is more likely} to belong to.

	\subsection*{Bayes' rule - a textbook-like example}

	There are 3 boxes, each containing 10 colored balls. In the first box ($A_1$) there are 8 blue balls and 2 red balls. In the second box ($A_2$) there are 3 blue balls and 7 red balls. There are only red balls in the third box ($A_3$). You have randomly chosen a box and pick up randomly one ball from it. The ball was blue ($Blue$). What is the probability that you got the first box ($A_1$)?
	
	\[P(A_1|B) = \frac{P(A_1)*P(Blue|A_1)}{P(A_1)*P(Blue|A_1)+P(A_2)*P(Blue|A_2)+P(A_3)*P(Blue|A_3)}\]
	\[= \frac{1/3*8/10}{1/3*8/10+1/3*3/10+1/3*0} = \frac{8/30}{8/30+3/30+0} = \frac{8}{11} = 0.73\]\\

	What is the probability that I got the second box ($A_2$)?
	
	\[P(A_2|Blue) = \frac{P(A_2)*P(Blue|A_1)}{P(A_1)*P(Blue|A_1)+P(A_2)*P(Blue|A_2)+P(A_3)*P(Blue|A_3)}\]
	\[= \frac{1/3*3/10}{1/3*8/10+1/3*3/10+1/3*0} = \frac{3/30}{8/30+3/30+0} = \frac{3}{11} = 0.27\]\\

	And the third box ($A_3$)?

	\[P(A_3|Blue) = \frac{P(A_2)*P(Blue|A_1)}{P(A_1)*P(Blue|A_1)+P(A_2)*P(Blue|A_2)+P(A_3)*P(Blue|A_3)}\]
	\[= \frac{1/3*0}{1/3*8/10+1/3*3/10+1/3*0} = 0\] 


	\subsection*{Bayes' rule - another way to think}
	Another to think about Bayes rule is as	a theorem that explains how we ought to change our’s initial/hypothetical/subjective probability in response to empirical evidence. Let's rearrenge our notation:\\
	
	T = theory or hypothesis \\

	E = represents a new piece of evidence that seems to confirm or disconfirm the theory.\\
	
	P(T) =  Prior Belief = Probability of (T). We can think of it as the expectation that our theory is correct \textbf{BEFORE} empirical investigation.\\
	
	P(T|E) = Posterior Probability. We can think of it as the expectation that our theory is correct \textbf{AFTER} empirical investigation.\\
	
	Then, we can use Bayes rule to compute the Posterior Probability of T being correct:\\
	
	\[P(T|E) = \frac{P(T)*P(E|T)}{P(T)*P(E|T)+P(T^c)*P(E|T^c)}=\]


\end{document}
